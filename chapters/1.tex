\section{Data semantics} % (fold)
\label{sec:data_semantics}
Il progetto viene svolto sul dataset (simulato) \textbf{
\href{https://www.kaggle.com/quentinvincenot/human-resources-analysis/data}{Human Resources Analytics}}. Questo
dataset contiene le informazioni sui dipendenti di un'azienda fittizia, suddivise in base alle colonne elencate
\begin{wraptable}{r}{8cm}
    \centering
    \begin{tabular}{| c | c |}
        \hline
        \textbf{Field} & \textbf{Type} \\ \hline
        satisfaction\_level & continuous \\ \hline
        last\_evaluation & continuous \\ \hline
        number\_project & discrete \\ \hline
        average\_montly\_hours & discrete \\ \hline
        time\_spend\_company & discrete \\ \hline
        work\_accident & discrete \\ \hline
        left & discrete \\ \hline
        promotion\_last\_5years & discrete \\ \hline
        sales & categorical \\ \hline
        salary & ordinal \\
        \hline
    \end{tabular}
    \caption{Colonne presenti nel Dataset e rispettivi tipi.}
    \label{col_tab}
\end{wraptable}
nella Tabella \ref{col_tab}. In questa tabella, per ogni colonna, è stato riportanto il tipo di dato usato per
classificarla, seguendo la nomenclatura fornita nel testo utilizzato durante il corso,
\textit{Guide to Intelligent Data Analysis}. Con la prima colonna, \textit{Satisfaction Level}, viene
fornita una valutazione quantitativa del livello di soddisfazione di ciascun dipendente, in un range che va da un
valore minimo di $0$ ad un massimo di $1$. \textit{Last Evaluation} fornisce il tempo trascorso, in anni,
dall'ultima valutazione delle performace del dipendente. \textit{Number Project} riporta il numero di progetti
completati da ciascun dipendente durante il periodo di lavoro. \textit{Average Montly Hours} rappresenta la media
delle ore di lavoro in un mese. \textit{Time Spend Company} corrisponde al numero di anni trascorsi
dal dipendente all'interno dell'azienda. \textit{Work Accident} esprime con un $1$ il coinvolgimento di un
dipendente in un incidente sul lavoro, altrimenti viene impostato come $0$. \textit{Left} è utilizzata per tener
traccia dei dipendenti che hanno lasciato l'azienda, per i quali viene usato il valore $1$, e per quelli che sono
rimasti, per i quali viene usato il valore $0$. \textit{Promotion last 5 year} esprime con un $1$ se il dipendente
è stato promosso negli ultimi $5$ anni, altrimenti assume il valore $0$. \textit{Sales} definisce il dipartimento
nel quale il dipendente lavora, e \textit{Salary} esprime il livello (\textit{low}, \textit{medium},
\textit{high}) nel quale rientra il salario del dipendente. \\
Le descrizioni delle colonne sono state estrapolate dai metadati forniti assieme al Dataset sulla pagina di Kaggle
\footnote{\href{https://www.kaggle.com/}{https://www.kaggle.com/}} nella quale il Dataset è contenuto.
% section data_semantics (end)
\section{Distribution of the variables and statistics} % (fold)
\label{sec:distribution_of_the_variables_and_statistics}
\begin{figure}
    \centering
    \begin{subfigure}{.4\linewidth}
        \includegraphics[scale=0.5]{images/salary.pdf}
        \caption{Distribuzione relativa alla colonna Salary.}
        \label{salary}
    \end{subfigure}
    \begin{subfigure}{.4\linewidth}
        \includegraphics[scale=0.5]{images/sales.pdf}
        \caption{Distribuzione relativa alla colonna Sales.}
        \label{sales}
    \end{subfigure}
    \begin{subfigure}{.4\linewidth}
        \includegraphics[scale=0.4]{images/satisfaction_level.pdf}
        \caption{Distribuzione relativa alla colonna Satisfaction Level.}
        \label{satisfaction_label}
    \end{subfigure}
    \begin{subfigure}{.4\linewidth}
        \includegraphics[scale=0.5]{images/average_montly_hours.pdf}
        \caption{Distribuzione relativa alla colonna Average Montly Hours.}
        \label{average_montly_hours}
    \end{subfigure}
\end{figure}
In questo paragrafo vengono presentati i grafici relativi alla distribuzione dei valori assunti dalle colonne
descritte nella sezione precedente. Viene omesso il codice con i quali tali grafici sono stati co{}struiti.
In Fig. \ref{salary} possiamo notare che la maggior parte degli impiegati che hanno deciso di lasciare l'azienda
aveva un salario rientrate nella categoria \textit{low}. Al contrario, un numero pressochè uguale di impiegati
all'interno dell'azienda ha un salatio rientrate nelle categorie \textit{low} e \textit{medium}. La Fig.
\ref{sales} rappresenta invece la distribuzione degli impiegati rispetto alla colonna Sales.
% section distribution_of_the_variables_and_statistics (end)
\section{Assessing data quality} % (fold)
\label{sec:assessing_data_quality}
\lipsum
% section assessing_data_quality (end)
\section{Variable transformations} % (fold)
\label{sec:variable_transformations}
\lipsum
% section variable_transformations (end)
\section{Pairwise correlations and eventual elimination of redundant variable} % (fold)
\label{sec:pairwise_correlations_and_eventual_elimination_od_redundant_variable}
\lipsum
% section pairwise_correlations_and_eventual_elimination_od_redundant_variable (end)
