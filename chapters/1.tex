\section{Obiettivi}
In questo progetto viene analizzato il dataset (simulato) \textbf{
  \href{https://www.kaggle.com/quentinvincenot/human-resources-analysis/data}{Human Resources Analytics}} che contiene
le informazioni sui dipendenti di un'azienda fittizia. Una parte rilevante dei lavoratori, circa il $X\%$ ha lasciato l'azienda

\todo[inline]{Il grafico è solo una prova, è da sistemare}

\begin{figure}[h]
  \centering
  \includegraphics[width=\textwidth]{images/obiettivi.pdf}
  \caption{Numero di lavoratori che hanno lasciato l'azienda o sono rimasti}
\end{figure}

I principali obiettivi di questo progetto sono:
\begin{itemize}
\item capire i motivi principali per cui un lavoratore ha lasciato l'azienda
\item prevedere se un lavoratore lascerà l'azienda o meno
\item indicare al management dell'azienda dei provvedimenti da prendere per ridurre il numero di impiegati che lasciano il lavoro  
\end{itemize}



\section{Data semantics} % (fold)
\label{sec:data_semantics}
Il progetto viene svolto sul dataset (simulato) \textbf{
\href{https://www.kaggle.com/quentinvincenot/human-resources-analysis/data}{Human Resources Analytics}}. Questo
dataset contiene le informazioni sui dipendenti di un'azienda fittizia, suddivise in base alle colonne elencate
\begin{wraptable}{r}{8cm}
    \centering
    \begin{tabular}{| c | c |}
        \hline
        \textbf{Field} & \textbf{Type} \\ \hline
        satisfaction\_level & continuous \\ \hline
        last\_evaluation & continuous \\ \hline
        number\_project & discrete \\ \hline
        average\_montly\_hours & discrete \\ \hline
        time\_spend\_company & discrete \\ \hline
        work\_accident & discrete \\ \hline
        left & discrete \\ \hline
        promotion\_last\_5years & discrete \\ \hline
        sales & categorical \\ \hline
        salary & ordinal \\
        \hline
    \end{tabular}
    \caption{Colonne presenti nel Dataset e rispettivi tipi.}
    \label{col_tab}
\end{wraptable}
nella Tabella \ref{col_tab}. In questa tabella, per ogni colonna, è stato riportanto il tipo di dato usato per
classificarla, seguendo la nomenclatura fornita nel testo utilizzato durante il corso,
\textit{Guide to Intelligent Data Analysis}. Con la prima colonna, \textit{Satisfaction Level}, viene
fornita una valutazione quantitativa del livello di soddisfazione di ciascun dipendente, in un range che va da un
valore minimo di $0$ ad un massimo di $1$. \textit{Last Evaluation} fornisce il tempo trascorso, in anni,
dall'ultima valutazione delle performace del dipendente. \textit{Number Project} riporta il numero di progetti
completati da ciascun dipendente durante il periodo di lavoro. \textit{Average Montly Hours} rappresenta la media
delle ore di lavoro in un mese. \textit{Time Spend Company} corrisponde al numero di anni trascorsi
dal dipendente all'interno dell'azienda. \textit{Work Accident} esprime con un $1$ il coinvolgimento di un
dipendente in un incidente sul lavoro, altrimenti viene impostato come $0$. \textit{Left} è utilizzata per tener
traccia dei dipendenti che hanno lasciato l'azienda, per i quali viene usato il valore $1$, e per quelli che sono
rimasti, per i quali viene usato il valore $0$. \textit{Promotion last 5 year} esprime con un $1$ se il dipendente
è stato promosso negli ultimi $5$ anni, altrimenti assume il valore $0$. \textit{Sales} definisce il dipartimento
nel quale il dipendente lavora, e \textit{Salary} esprime il livello (\textit{low}, \textit{medium},
\textit{high}) nel quale rientra il salario del dipendente. \\
Le descrizioni delle colonne sono state estrapolate dai metadati forniti assieme al Dataset sulla pagina di Kaggle
\footnote{\href{https://www.kaggle.com/}{https://www.kaggle.com/}} nella quale il Dataset è contenuto.
% section data_semantics (end)
\section{Distribution of the variables and statistics} % (fold)
\label{sec:distribution_of_the_variables_and_statistics}
In questo paragrafo vengono presentati i grafici relativi alla distribuzione dei valori assunti dalle colonne
descritte nella sezione precedente. Viene omesso il codice con i quali tali grafici sono stati costruiti. Ogni
figura viene presentata accompagnata da una descrizione della distribuzione rappresentata. \newpage
\begin{figure}[H]
    \captionsetup[subfigure]{font=scriptsize,labelfont=scriptsize}
    \centering
    \subcaptionbox{Tra i $3571$ impiegati che hanno lasciato l'azienda, $2172$ rietrano nella categoria salariale
    \textit{low}, mentre tra gli $11428$ impiegati rimasti nell'azienda, $5144$ rientrano nella categoria salariale
    \textit{low}.\label{salary}}{\includegraphics[scale=0.4]{images/salary.pdf}}
    \subcaptionbox{Come possiamo osservare, il reparto \textit{sales} è quello con il più alto numero di impiegati,
    sia che essi abbiano lasciato l'azienda sia che siano rimasti.\label{sales}}
    {\includegraphics[scale=0.4]{images/sales.pdf}}
    \subcaptionbox{Per gli impiegati che hanno lasciato l'azienda, la \textbf{media} del livello di soddisfazione è
    $0.44$, la \textbf{deviazione standard} è $0.26$, i valori \textbf{minimo} e \textbf{massimo} sono,
    rispettivamente, $0.09$ e $0.92$. Per quanto riguarda gli impiegati rimasti abbiamo invece una \textbf{media}
    pari a $0.67$, una \textbf{deviazione standard} pari a $0.22$ e un valore \textbf{minimo} e \textbf{massimo}
    pari a $0.12$ e $1.0$.\label{satisfaction_level}}{\includegraphics[scale=0.5]{images/satisfaction_level.pdf}}
    \subcaptionbox{Per gli impiegati che hanno lasciato l'azienda, la \textbf{media} delle ore di lavoro mensili è
    $207.42$, la \textbf{deviazione standard} è $61.20$, i valori \textbf{minimo} e \textbf{massimo} sono,
    rispettivamente, $126$ e $310$. Per quanto riguarda gli impiegati rimasti abbiamo invece una \textbf{media}
    pari a $199.06$, una \textbf{deviazione standard} pari a $45.68$ e un valore \textbf{minimo} e
    \textbf{massimo} pari a $96$ e $287$.\label{average_montly_hours}}
    {\includegraphics[scale=0.5]{images/average_montly_hours.pdf}}
    \caption{Da sinistra verso destra, e dall'alto verso il basso, sono raffigurate le distribuzioni delle
    colonne \textit{Salary}, \textit{Sales}, \textit{Satisfaction Level} e \textit{Average Montly Hours}.
    In rosso vengono rappresentati gli impiegati che sono rimasti nell'azienda, mentre in blu quelli che se ne
    sono andati. Per ogni colonna vengono fornite delle informazioni riguardo alla \textbf{media}, alla
    \textbf{deviazione standard} e ai valori \textbf{massimi} e \textbf{minimi} dei parametri analizzati.}
\end{figure}
\begin{figure}[H]
    \captionsetup[subfigure]{font=scriptsize,labelfont=scriptsize}
    \centering
    \subcaptionbox{Come possiamo osservare, la maggior parte degli impiegati che hanno lasciato l'azienda non ha
    avuto una promozione negli ultimi $5$ anni. Soltato $19$ impiegati su $3571$ sono stati promossi. Lo stesso
    vale per gli impiegati rimasti, ma ovviamente in proporzioni diverse.\label{promotion_last_5_years}}
    {\includegraphics[scale=0.4]{images/promotion_last_5_years.pdf}}
    \subcaptionbox{La maggior parte degli impiegati che hanno lasciato l'azienda ha svolto soltato $2$ progetti,
    mentre quelli che sono rimasti hanno svolto per la maggior parte $3$ progetti, e in misura leggermente minore
    $4$.\label{number_project}}
    {\includegraphics[scale=0.4]{images/number_project.pdf}}
    \subcaptionbox{Gli impiegati che se ne sono andati avevano trascorso per la maggior parte $3$ anni all'interno
    dell'azienda, allo stesso modo gran parte degli impiegati al momento in azienda lavora al suo interno da $3$
    anni.\label{time_spend_company}}
    {\includegraphics[scale=0.4]{images/time_spend_company.pdf}}
    \subcaptionbox{Soltanto $169$ impiegati tra i $3571$ che se ne sono andati ha avuto un incidente sul lavoro,
    mentre gli impiegati ancora all'interno dell'azienda ad aver subito un incidente sono $2000$ su $11428$
    \label{work_accident}}{\includegraphics[scale=0.4]{images/work_accident.pdf}}
    \subcaptionbox{\label{last_evaluation}}{\includegraphics[scale=0.5]{images/last_evaluation.pdf}}
    \caption{Da sinistra verso destra, e dall'alto verso il basso, sono raffigurate le distribuzioni delle
    colonne \textit{Promotion Last 5 Years}, \textit{Number Project}, \textit{Time Spend Company},
    \textit{Work Accident} e \textit{Last Evaluation}. In rosso vengono rappresentati gli impiegati che sono
    rimasti nell’azienda, mentre in blu quelli che se ne sono andati. Per le prime quattro visualizzazioni
    sono fornite delle descrizioni sommarie, trattandosi di dati di tipo discreto.}
\end{figure}
% section distribution_of_the_variables_and_statistics (end)
\section{Assessing data quality} % (fold)
\label{sec:assessing_data_quality}
\lipsum
% section assessing_data_quality (end)
\section{Variable transformations} % (fold)
\label{sec:variable_transformations}
\lipsum
% section variable_transformations (end)
\section{Pairwise correlations and eventual elimination of redundant variable} % (fold)
\label{sec:pairwise_correlations_and_eventual_elimination_od_redundant_variable}
\lipsum
% section pairwise_correlations_and_eventual_elimination_od_redundant_variable (end)
