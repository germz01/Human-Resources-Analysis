\section{Conclusioni} % (fold)
\label{sec:conclusion}
L'analisi portata avanti ha dato i seguenti risultati:
\\L'azienda ha dipendenti che rimangono in media pochi anni a lavorare nella loro società, la nostra analisi verteva sul trovare i principali motivi di questo evento e suggerire quindi all'azienda dei consigli per far sì che il dipendente non lasci l'azienda prematuramente. Ovviamente è nell'interesse dell'azienda che rimangano a lavorare in essa i dipendenti più abili e competenti nel proprio lavoro. \\Dall'analisi risalgono però anche vari dipendenti che hanno lasciato prematuramente l'azienda in quanto non ritenuti abbastanza competenti dall'azienda stessa, visto lo scarso valore come valutazione. \\Noi invece ci soffermiamo su quelli che nonostante ritenuti meritevoli hanno comunque deciso di lasciare l'azienda. Ci sono gruppi di dipendenti che lavorano molte ore di lavoro e realizzano vari progetti che hanno un livello di soddisfazione molto basso che sono andati via. Il numero invece degli incidenti per l'analisi proposta sembra non incidere sull'uscita prematura del dipendente dall'azienda. \\L'analisi della classificazione con le association rules ha fatto emergere che ci sono tutt'ora dipendenti all'interno della azienda che si trovano in una situazione simile a dipendenti che hanno già lasciato l'azienda. Dal classificatore risultano $810$ ma si suppone in via teorica, come anche definito nella sezione apposita che questo sia solo un valore indicativo, non quello reale che si suppone essere anche più elevato. 
\\Detto ciò è emerso inoltre dalle analisi che le problematiche consistenti dell'azienda sono il numero elevato di ore di lavoro, per cui si consiglia di ridurle ad un orario medio di ore mensili di un dipendente standard. Inoltre si dovrebbero cercare di stimolare i dipendenti più abili e meritevoli, fedeli all'azienda da anni con delle promozioni. Un fatto che è risaltato infatti è che all'interno dell'azienda sono stati promossi solo $319$ impiegati , ovvero neanche il $2.25\%$ del totale dei dipendenti analizzati. 

