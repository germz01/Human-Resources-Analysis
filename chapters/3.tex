Per definire le association rules prima si sono seguite le seguenti operazioni:
\begin{itemize}
\item Abbiamo trasformato le variabili da stringhe a numeriche, per quanto riguarda gli attributi \textit{Salary}
e \textit{Department}.
\item Abbiamo raggruppato le variabili \textit{Last Evaluation}, \textit{Satisfaction Level} e
\textit{Average Montly Hours}, usando $3$ bins sia per la prima che per la seconda variabile, usando intervalli
specifici, e $2$ bins per la terza variabile, applicando anche in questo caso una suddivisione ad hoc.
\item Per rendere unici i risultati numerici, è stata aggiunta una stringa subito successiva al valore numerico
in modo da non renderlo ambiguo e sopratutto in modo da poter capire univocamente a che attributo si riferisca.
\end{itemize}
\todo[inline]{Aggiungere la divisione in intervalli scelta e legenda con le abbreviazioni, oppure sotto non abbreviare}
\section{Frequent patterns extraction with different values of support and different types} % (fold)
\label{sec:frequent_patterns_extraction_with_different_values_of_support_and_different_types}
Dopo aver eseguito i passi preliminari descritti precedentemente abbiamo svolto l'analisi inerente ai
\textit{frequent patterns} attraverso l'applicazione dell'algoritmo \textit{Apriori}. Per ogni iterazione
dell'algoritmo, abbiamo considerato, indipendentemente dal \textit{support}, gli itemsets con $2$ o più items al
loro interno. Inoltre, al fine di avere una panoramica più completa, abbiamo svolto l'analisi per i
\textit{frequent itemsets}, per i \textit{closed frequent itemsets} e per i \textit{maximal frequent itemsets}.
Abbiamo quindi cominciato l'analisi con un support pari a $20$, ossia prendendo in considerazione soltanto gli
itemsets presenti in almeno il $20\%$ della transazioni. Successivamente abbiamo utilizzato un support pari a
$30$. Le quantità di frequent itemsets scoperte al variare dei paramentri sono riportate in Tabella
\ref{tab:freq_items}.
\begin{table}[H]
    \centering
    \resizebox{0.8\textwidth}{!}{
        \begin{tabular}{| c | c | c | c |}
            \hline
            \textbf{Support Threshold} & \textbf{Frequent Itemsets} & \textbf{Closed Frequent Itemsets} &
            \textbf{Maximal Frequent Itemsets} \\ \hline
            $20$ & $137$ & $130$ & $30$ \\ \hline
            $30$ & $46$ & $45$ & $11$ \\ \hline
        \end{tabular}
    }
    \caption{Quantità di frequent itemsets trovati per ogni tipologia e support utilizzati durante l'analisi.}
    \label{tab:freq_items}
\end{table}
Come era lecito aspettarsi, esiste un rapporto di proporzionalità inversa tra la soglia di support e il numero di
frequent itemsets scoperti.
% section frequent_patterns_extraction_with_different_values_of_support_and_different_types (end)
\section{Discussion of the most interesting frequent patterns} % (fold)
\label{sec:discussion_of_the_most_interesting_frequent_patterns}
Passiamo adesso alla descrizione dei frequent items più interessanti che sono stati scoperti durante l'analisi. In
Tabella \ref{tab:interesting:freq_items} vengono riportati gli itemsets più interessanti dal punto di vista del
supporto pari a $20$ scoperti durante l'analisi.
\begin{table}[H]
    \centering
    \resizebox{\textwidth}{!}{
        \begin{tabular}{| c | c | c | c | c | c |}
            \hline
            \textbf{Frequent Itemsets ($ST = 20$)} & \textbf{Support} &
            \textbf{Closed Frequent Itemsets ($ST = 20$)} & \textbf{Support} &
            \textbf{Maximal Frequent Itemsets ($ST = 20$)} & \textbf{Support}\\
            \hline
            (N\_WA, \ N\_P)  & $0.84$ & (N\_WA, \ N\_P)  & $0.84$ & (standard\_H, N\_L, N\_WA, N\_P) & $0.31$ \\
            \hline
            (N\_L, \ N\_P) & $0.74$ & (N\_L, \ N\_P) & $0.74$ & (intensive\_H, N\_L, N\_WA, N\_P) & $0.30$ \\
            \hline
            (N\_L, \ N\_WA) & $0.63$ & (N\_L, \ N\_WA) & $0.63$ & (0\_S, \ N\_L, \ N\_WA, \ N\_P) & $0.28$ \\
            \hline
            (N\_L, \ N\_WA, \ N\_P) & $0.61$ & (N\_L, \ N\_WA, \ N\_P) & $0.61$ & (1\_S, N\_L, N\_WA, N\_P) &
            $0.28$ \\ \hline
        \end{tabular}
    }
    \caption{Frequent itemsets con supporto maggiore scoperti durante l'analisi utilizzando un supporto pari a
    $20$. Con N\_WA intendiamo l'item relativo all'assenza di incidenti sul lavoro, con N\_P l'item relativo alla
    mancanza di promozioni, con N\_L l'item relativo ai dipendenti ancora in azienda, con 0\_S l'item relativo ai
    dipendenti con salario minimo, con 1\_S l'item relativo ai dipendenti con salario medio, con intensive\_H
    intendiamo i dipendenti con un quantitativo di ore mensili compreso tra $200$ e $300$ e con standard\_H
    intendiamo i dipendenti con un quantitativo di ore mensili inferiore a $200$.}
    \label{tab:interesting:freq_items}
\end{table}
Descriviamo per primi i frequent itemsets e i closed frequent itemsets, visto che sono identici. Possiamo notare
come la situazione presentata proponga in maggioranza impiegati i quali non hanno subito incidenti sul lavoro,
che non sono stati promossi e che non hanno lasciato l'azienda. Per quanto riguarda i maximal frequent itemsets
troviamo che gli impiegati con carichi di lavoro sia standard che elevati, che non hanno lasciato l'azienda, non
hanno avuto incidenti sul lavoro e che non sono stati promossi negli ultimi $5$ anni sono i più diffusi, seguiti
dagli impiegati di salario minimo e medio, non promossi e i quali non hanno avuto incidenti sul lavoro.
Portando la soglia del support a $30$, gli itemsets più diffusi sono gli stessi che sono  stati descritti per la
soglia pari a $20$, evitiamo quindi di descriverli.
% section discussion_of_the_most_interesting_frequent_patterns (end)
\section{Association rules extraction with different values of confidence }

\begin{itemize}
\item Regole "generali'':
inizialmente sono state cercate associazioni interessanti valide per un numero ampio di impiegati, fissando un alto supporto minimo, pari a $MinSupp = 20\%$, nella ricerca degli itemset frequenti con l'algoritmo \textit{apriori}.
\item Regole "specifiche'':
oltre alla ricerca di regole abbastanza generali (con ampio supporto) si è considerato che uno degli obiettivi principali delle analisi contenute in questo report è capire il perchè una parte consistente dei dipendenti ha lasciato l'azienda. 
La percentuale di dipendenti che hanno lasciato l'azienda corrisponde al $24\%$ del totale, considerando come significativa una regola che riguardi almeno il $20\%$ dei dipendenti che hanno lasciato l'azienda, risulta un supporto minimo pari a circa $MinSupp=5\%$.
\end{itemize}

L'algoritmo a priori è stato eseguito dunque per entrambi i valori di $MinSupp$ indicati, variando la confidenza minima \textit{MinConf} per valori compresi tra $50-100\%$. In Fig. sono riportate il numero di regole ottenute, filtrate per valori di $Lift>1$, in funzione della confidenza.
Le regole cercate con l'algoritmo utilizzato hanno un solo item come parte conseguente, per facilitare l'analisi e perchè...


\begin{figure}[htbp]
  \centering
  \includegraphics[width=0.7\textwidth]{../images/rules/andamentonumero.pdf}  \caption{Grafico,mettere LEGENDA PER COLORE E SIZE (Support)}
  \label{fig:rules-number}
\end{figure}


\begin{minipage}[c]{0.5\textwidth}
\begin{figure}[H]
  \centering
  \includegraphics[width=\textwidth]{../images/rules/rules_supp20.pdf}
  \caption{Grafico,mettere LEGENDA PER COLORE E SIZE (Support)}
  \label{fig:rules_supp20}
\end{figure}
\end{minipage}
\begin{minipage}[c]{0.5\textwidth}
\begin{figure}[H]
  \centering
  \includegraphics[width=\textwidth]{../images/rules/rules_supp5.pdf}
  \caption{Grafico,mettere LEGENDA PER COLORE E SIZE (Support)}
  \label{fig:rules_supp5}
\end{figure}
\end{minipage}

In Fig. \ref{fig:rules_supp20} si osserva che la maggior parte delle regole ha un Lift basso, prossimo ad uno.
Le regole oggettivamente interessanti, con $Lift>2$, ben visibili nella parte alta del grafico sono riportate in Tab REF.
e mostrano una correlazione positiva tra un alto livello di valutazione, \textit{very good, Last Evaluation} ed un alto grado di soddisfazione degli impiegati, a cui si accompagnano nella parte antecedente la mancanza di incidenti sul lavoro (\textit{N WA}) ed il fatto di essere rimasti in azienda. Queste regole non sono particolarmente interessanti dal punto di vista dell'analisi, poichè abbastanza ovvie, e non aggiungono molto alle
analisi statistiche eseguite, inoltre il valore di $Lift$ non è così ampio. (è confrontabile con qualcosa? valori di riferimento?)

Usando $MinSupp=5\%$ naturalmente il numero di associazioni trovate è notevolmente superiore, e si possono osservare in Fig. REF regole con indice di $Lift$ che arriva a valori massimi di 6-7. Per valori del supporto superiori al $15\%$ il Lift è prossimo ad 1, tranne che per le regole discusse precedentemente.

\todo[inline]{Il problema che rimane da affrontare è il seguente: Per ottenere delle regole per le persone he hanno lasciato serve un supporto basso, (che ho fissato al 5$\%$), ma questo comporta un numero elevato di regole. Come filtrarle? Sto pensando a questo, ho provato in diversi modi, ma non sono ancora giunto ad una conclusione. Al momento ci sono molti grafici, ma alcuni andranno tolti, sono ancora qua per supporto all'analisi.}


\begin{table}[h!]
  \centering
  \scalebox{0.7}{
\input{data/rules/rules_supp20.txt}  
}
\caption{Association rules per}
\end{table}

\begin{figure}[htbp]
  \centering
  \includegraphics[width=1\textwidth]{../images/rules/left_scatter.pdf}
  \caption{Grafico,mettere LEGENDA PER COLORE E SIZE (Support)}
  \label{fig:left_scatter_}
\end{figure}



\subsection{Regole per impiegati che hanno lasciato (sistemare titolo)}

\todo[inline]{SCRIVERE LA SPIEGAZIONE, tabelle successive sono poi da togliere}
Tra le regole venute fuori con $MinSupp=5$ ho estratto quelle che hanno come parte conseguente
gli impiegati che hanno lasciato, le ho poi filtrate per alta confidence, $Conf>0.90$,
ottenendo circa 80 regole, per rappresentarle e non elencarle tutte ho fatto il seguente grafico,
in cui è rappresentato il numero di regole in cui compare ciascun item nella parte antecedente.
SPIEGARE MEGLIO, DA FINIRE


\begin{figure}[htbp]
  \centering
  \includegraphics[width=0.7\textwidth]{../images/rules/rules_left.pdf}
  \caption{Grafico,mettere LEGENDA PER COLORE E SIZE (Support)}
  \label{fig:rules_left}
\end{figure}



\begin{table}[h!]
  \centering
  \scalebox{0.7}{
\input{data/rules/rules_left.txt}  
}
\caption{Association rules per}
\end{table}


\begin{table}[h!]
  \centering
  \scalebox{0.7}{
\input{data/rules/rules_supp5.txt}  
}
\caption{Association rules per}
\end{table}


\clearpage
\newpage
\section{Discussion of the most interesting rules }
Abbiamo inoltre rilevato delle rules che nonostante non siano frequenti sono comunque interessanti per l'analisi che stiamo portando avanti. Queste sono:
-
-
Da queste possiamo ricavare che...






\section{Use the most meaningful rules to replace missing values and evaluate the accuracy}
Nel nostro dataset non sono presenti missing values quindi non è stata necessaria la valutazione delle più significative rules e la valutazione dell'accuratezza per rimpiazzare questi.


\section{Use the most meaningful rules to predict if an employee will leave prematurely or not and evaluate the accuracy }
Date le varie association rules trovate dalle varie prove queste sono quelle più significative per predirre se un impiegato lascerà prematuramente l'azienda oppure no:

- AR ... accuratezza trovata:
- AR1
- AR2

Da queste possiamo scaturire che un impiegato lasci il posto di lavoro prematuramente quando è nelle seguenti condizioni:

Invece rimarrà quando avrà una condizione del tipo:







