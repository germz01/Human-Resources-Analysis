\section{Obiettivi}
%%\begin{comment}
\begin{wrapfigure}[10]{r}{0.3\textwidth}
  \centering
  \includegraphics[height=4.5cm]{obiettivi.pdf}
  \caption{Numero di lavoratori}
  \label{fig:obiettivi}
\end{wrapfigure}
%%\end{comment}
\begin{comment}
\begin{figure}[H]
  \centering
  \includegraphics[height=5cm]{obiettivi.pdf}
  \caption{Numero di lavoratori}
  \label{fig:obiettivi}
\end{figure}
\end{comment}

In questo progetto viene analizzato il dataset (simulato) \textit{\href{https://www.kaggle.com/quentinvincenot/human-resources-analysis/data}{Human Resources Analytics}} contenente le informazioni sui dipendenti di un'azienda fittizia. Come mostrato in Figura \ref{fig:obiettivi} su un totale di $14999$ dipendenti il $24\%$, corrispondente a $3571$ lavoratori, ha lasciato l'azienda. Gli obiettivi primari dell'analisi sono i seguenti:

\begin{itemize}[noitemsep]
\item capire i motivi principali per cui i lavoratori hanno lasciato l'azienda;
\item predire probabilisticamente se un lavoratore lascerà in futuro l'azienda;
\item indicare al management dell'azienda dei provvedimenti da attuare per ridurre il numero di impiegati che la abbandonano.
\end{itemize}

\section{Data semantics} % (fold)
\label{sec:data_semantics}

Il dataset è composto da 10 variabili relative ai dipendenti dell'azienda, riportate in tabella \ref{tab:variabili}, delle quali 5 sono di tipologia categorica, di cui una ordinale, e 5 di tipologia numerica.

\begin{wraptable}{r}{8cm}
    \centering
    \begin{tabular}{ c  c }
\toprule
        \textbf{Variable} & \textbf{Type} \\ 
\midrule
        Left & categorical \\
        Department & categorical \\ 
        Promotion\_last\_5years & categorical \\ 
        Work\_accident & categorical \\ 
        Salary & ordinal \\ 
\midrule
        Satisfaction\_level & continuous \\ 
        Last\_evaluation & continuous \\ 
        Average\_montly\_hours & discrete \\ 
        Time\_spend\_company & discrete \\
        Number\_projects & discrete \\ 
\bottomrule
    \end{tabular}
    \caption{Variabili presenti nel Dataset e rispettivi tipi.}
    \label{tab:variabili}
\end{wraptable}

La variabile \textit{Left} suddivide il dataset tra i dipendenti che hanno lasciato l'azienda e quelli che ci lavorano attualmente, associando alle rispettive categorie i valori $1$ e $0$.
I dipendenti lavorano in 10 diversi dipartimenti indicati nella variabile \textit{Department}, che è stata rinominata
rispetto all'originale \textit{Sales} per chiarezza semantica.
La promozione o meno di un dipendente durante gli ultimi $5$ anni è espressa dalla variabile 
\textit{Promotion last 5 Year} con un $1$ in caso positivo e con $0$ altrimenti.
\textit{Work Accident} indica con un $1$ il coinvolgimento di un dipendente in un incidente sul lavoro,
e con $0$ il caso contrario. \textit{Salary} esprime il livello (\textit{low}, \textit{medium}, \textit{high}) nel quale rientra il salario
del dipendente. Con la variabile \textit{Satisfaction Level} viene fornita una valutazione quantitativa del livello di
soddisfazione di ciascun dipendente, in un range che va da un valore minimo di $0$ ad un massimo di $1$.
\textit{Last Evaluation} fornisce l'ultima valutazione riguardo le performance del
dipendente, compresa tra $0$ ed $1$. \textit{Average Montly Hours} rappresenta la media delle ore di lavoro in un mese mentre
\textit{Time Spend Company} corrisponde al numero di anni trascorsi dal dipendente all'interno dell'azienda.
\textit{Number Projects} riporta il numero di progetti completati da ciascun dipendente durante il
periodo di lavoro. Le descrizioni delle variabili sono state estrapolate dai metadati forniti assieme al dataset
sulla pagina di Kaggle \footnote{\href{https://www.kaggle.com/}{https://www.kaggle.com/}} nella quale il
dataset è pubblicato.
\begin{comment}
La scelta di catalogare le colonne \textit{Work Accident}, \textit{Left} e
\textit{Promotion Last 5 Years} come variabili di tipo categorico verrà opportunamente motivata nelle s
ezioni successive. 
Direi che non c'è da motivarlo ulteriormente
\end{comment}
%% section data_semantics (end)
\section{Distribution of the variables and statistics} %% (fold)
\label{sec:distribution_of_the_variables_and_statistics}
In questo paragrafo vengono presentati i grafici relativi alla distribuzione dei valori assunti dalle variabili
descritte nella sezione precedente. Per dare una migliore interpretazione a questi abbiamo deciso di sostenere
un'analisi accurata che contraddistingue i dipendenti che lavorano nell'azienda, rappresentati dal colore blu, e
quelli che invece la hanno lasciata, rappresentati dal colore rosso.\\
Prima di tutto vogliamo studiare la  distribuzione dei dipendenti rispetto alle variabili
categoriche escludendo la distribuzione dei dipendenti rispetto a left in quanto già esplicata in precedenza nella sezione Obiettivi. 

\begin{minipage}{8.1cm}
   \centering
    \begin{figure}[H]
        \includegraphics[width=\textwidth]{images/newplottedvariables/sales.pdf}
        \caption{Distribuzione relativa alla variabile \textit{Sales}}
        \label{fig:sales}
\end{figure}
\end{minipage}
\begin{minipage}{8.1cm}
\begin{table}[H]
        \centering
        \begin{tabular}{| c | c | c | c |} \hline
            \textbf{Dipartimento} & \textbf{InAzienda} & \textbf{OutAzienda} \\ \hline
            1  & & \\ \hline
            1  & & \\ \hline
            1  & & \\ \hline
            1  & & \\ \hline
            1  & & \\ \hline
            1  & & \\ \hline
            1  & & \\ \hline
        \end{tabular}
        \label{tab:salary}
	 \caption{Distribuzione salario per dipendente.}
\end{table}
\end{minipage}

La distribuzione mostrata nel grafico indica la distribuzione degli impiegati nei
vari settori, questa dati sono esplicati nella tabella adiacente al grafico. \\
\begin{minipage}{8.1cm}
   \centering
    \begin{figure}[H]
       \includegraphics[width=\textwidth]{images/newplottedvariables/work_accident.pdf}
        \caption{Distribuzione relativa alla variabile \textit{Work Accident}}
        \label{fig:work_accident}
\end{figure}
\end{minipage}
\begin{minipage}{8.1cm}
\centering
    \begin{figure}[H]
        \includegraphics[width=\textwidth]{images/newplottedvariables/promotion_last_5_years.pdf}
        \caption{Distribuzione relativa alla variabile \textit{Promotion Last 5 Years}}
\label{fig:promotion}
\end{figure}
\end{minipage}

Nel grafico a sinistra si studia il rapporto tra i dipendenti e la presenza o meno di un
infortunio durante il periodo di lavoro all'interno dell'azienda e si è riscontrato che soltanto $169$ impiegati
tra i $3571$ che se ne sono andati hanno avuto un incidente sul lavoro, mentre gli impiegati ancora all'interno
dell'azienda ad aver subito un incidente sono $2000$ su $11428$. \\
Nel grafico a destra invece rapportiamo ciascun dipendente al fatto che questo sia stato
promosso negli ultimi 5 anni oppure no, possiamo ricavare una informazione importante, la totalità degli
impiegati che hanno lasciato l'azienda non ha avuto una promozione negli ultimi $5$ anni. Soltato $19$ impiegati
su $11428$ di quelli rimasti sono stati promossi.\\
Una volta studiate le distribuzioni categoriche continuiamo l'analisi con gli altri attributi, per renderle più
esplicative abbiamo affiancato a ciascun grafico la tabella che lo descrive in modo accurato e in modo che fosse
autoesplicativo. \\ 
Abbiamo rapportato in primis i dipendenti con il proprio salario:\\
\begin{minipage}{8.1cm}
   \centering
    \begin{figure}[H]
        \includegraphics[width=\textwidth]{images/newplottedvariables/salary.pdf}
        \caption{Distribuzione relativa alla variabile \textit{Salary}}
        \label{fig:salary}
\end{figure}
\end{minipage}
\begin{minipage}{8.5cm}
\begin{table}[H]
        \centering
        \begin{tabular}{| c | c | c |} \hline
                   \textbf{Dipendente} & \textbf{InAzienda} & \textbf{OutAzienda}  \\ \hline
            Low & 5144 ($\sim$45.02\%) & 2172 ($\sim$60.8\%) \\ \hline
            Medium & 5129 ($\sim$44.88\%)  & 1317 ($\sim$36.9\%) \\ \hline
           High & 1155 ($\sim$10.1\%)  & 82 ($\sim$ 2.3\%) \\ \hline
        \end{tabular}
        \caption{Distribuzione salario per dipendente.}
        \label{tab:salary}
\end{table}
\end{minipage}
\\Le percentuali che vengono indicate nella tabella  non sono in base alla totalità dei dipendenti ma riguardano solo il tipo di
dipendente definito dalla riga di appartenenza.\\
Proseguendo poi con il rapporto tra dipendenti ancora in azienda e non, e il numero delle ore di lavoro in media:
\begin{minipage}{8.3cm}
   \centering
    \begin{figure}[H]
        \includegraphics[width=\textwidth]{images/newplottedvariables/average_montly_hours.pdf}
        \caption{Distribuzione relativa alla variabile \textit{Average Montly Hours}}
        \label{fig:average_montly_hours}
\end{figure}
\end{minipage}
\begin{minipage}{8.5cm}
\begin{table}[H]
        \centering
        \begin{tabular}{| c | c | c |} \hline
                   \textbf{Dipendente} & \textbf{InAzienda} & \textbf{OutAzienda}  \\ \hline
            Media & 199.06 & 207.42 \\ \hline
            Dev.std. & 45.68	& 61.20 \\ \hline
            min,max & 96 , 287  &  126 , 310	\\ \hline
        \end{tabular}
        \label{tab:avg_work}
\end{table}
\end{minipage}
\\\\Dipendenti left e non rispetto al Numero di progetti:\\
\begin{minipage}{8.3cm}
   \centering
    \begin{figure}[H]
         \includegraphics[width=\textwidth]{images/newplottedvariables/number_project.pdf}
        \caption{Distribuzione relativa alla variabile \textit{Number Project}}
        \label{fig:num_proj}
\end{figure}
\end{minipage}
\begin{minipage}{8.5cm}
\begin{table}[H]
        \centering
        \begin{tabular}{| c | c | c |} \hline
                 \textbf{Dipartimento} & \textbf{InAzienda} & \textbf{OutAzienda} \\ \hline
            1  & & \\ \hline
            1  & & \\ \hline
            1  & & \\ \hline
            1  & & \\ \hline
            1  & & \\ \hline
            1  & & \\ \hline
            1  & & \\ \hline
        \end{tabular}
        \label{tab:salary}
\end{table}
\end{minipage}
\\\\\\A questo punto è giusto analizzare il livello di soddisfazione dei dipendenti presente e quello della ultima valutazione:

\begin{minipage}{8.3cm}
   \centering
    \begin{figure}[H]
        \includegraphics[width=\textwidth]{images/newplottedvariables/satisfaction_level.pdf}
        \caption{Distribuzione relativa alla variabile \textit{Satisfaction Level}}
        \label{fig:sat_lev}
\end{figure}
\begin{table}[H]
        \centering
        \begin{tabular}{| c | c | c |} \hline
                   \textbf{Dipendente} & \textbf{InAzienda} & \textbf{OutAzienda}  \\ \hline
            Media & 0.67 & 0.44 \\ \hline
            Dev.std. & 0.22 & 0.26\\ \hline
            min,max & 0.12 , 1.0  &  0.09 , 0.92 \\ \hline
        \end{tabular}
        \label{tab:sat_lev}
\end{table}
\end{minipage}
\begin{minipage}{8.3cm}
 \centering
    \begin{figure}[H]
        \includegraphics[width=\textwidth]{images/newplottedvariables/last_evaluation.pdf}
        \caption{Distribuzione relativa alla variabile \textit{Last Evaluation}}
        \label{fig:last_ev}
\end{figure}
\begin{table}[H]
        \centering
        \begin{tabular}{| c | c | c |} \hline
                   \textbf{Dipendente} & \textbf{InAzienda} & \textbf{OutAzienda}  \\ \hline
            Media & 0.67 & 0.44 \\ \hline
            Dev.std. & 0.22 & 0.26\\ \hline
            min,max & 0.12 , 1.0  &  0.09 , 0.92 \\ \hline
        \end{tabular}
        \label{tab:sat_lev}
\end{table}
\end{minipage}
\\

% section distribution_of_the_variables_and_statistics (end)
\section{Data quality} % (fold)
\label{sec:assessing_data_quality}

- Missing values

- Outliers

L'individuazione dei possibili outliers di una variabile numerica consiste nel verificare se siano presenti dei valori estremi rispetto alla distribuzione dei dati osservati. I test comunemente utilizzati, come il test di Grubb o il criterio di Chauvenet, sono basati sull'assunzione di una distribuzione di probabilità gaussiana, che non si osserva per le variabili numeriche del dataset analizzato (spiegare in distribution of the variables).
Un metodo robusto e di immediata applicazione è quello di osservare il boxplot dei dati, identificando come candidati outliers i valori che si trovano al di fuori dei whiskers, ovvero valori $x$ della variabile osservata per cui $|x-\tilde{x}|>2\,IQR(x)$, dove $\tilde{x}$ è la mediana ed $IQR(x)$ lo scarto interquartile.

\begin{figure}[ht]
  \centering
\includegraphics[width=\textwidth]{images/boxplots.pdf}
  \caption{Boxplots per le variabili numeriche}
  \label{fig:boxplots}
\end{figure}




% section assessing_data_quality (end)
\section{Variable transformations} % (fold)
\label{sec:variable_transformations}
Analizzando il significato delle variabili presenti nel dataset, abbiamo deciso di rappresentare
\textit{Work Accident} e \textit{Left} utilizzando il tipo categorico piuttosto che quello discreto. Questa
scelta è stata motivata dall'analisi semantica delle due variabili, le quali forniscono una risposta del tipo
"Sì o No" alle domande relative agli incidenti sul lavoro e all'abbandono o meno dell'azienda da parte dei
dipendenti.

% section variable_transformations (end)
\section{Pairwise correlations and eventual elimination of redundant variable} % (fold)
\label{sec:pairwise_correlations_and_eventual_elimination_od_redundant_variable}
\begin{figure}[b!]
    \centering
    \includegraphics[width=0.6\textwidth]{images/correlation_matrix.pdf}
    \caption{Correlation Matrix delle variabili presenti nel Dataset.}
    \label{fig:pair_corr}
\end{figure}
In questa sezione abbiamo studiato la correlazione ovvero la relazione lineare tra i vari attributi continuous
o discreti.
Dalla matrice riportata in Figura \ref{fig:pair_corr} possiamo rilevare se ci sia una correlazione positiva,
nulla o negativa. Sia per quanto riguarda la correlazione positiva sia per quella negativa si caratterizzano in
settori: con valori da $0$ a $0.3$ correlato debolmente, da $0.3$ a $0.7$ moderatamente o maggiore di $0.7$
fortemente (rispettivamente per la negativa i segni saranno negativi).
Da questo possiamo definire che ad avere una correlazione debole è la variabile time\_spend\_company con
left, last\_evaluation, number\_project e average\_monthly\_hours. Queste ultime, ad eccezione di left,
invece sono correlate fra loro in modo moderato con un valore massimo di $0.42$ tra average\_monthly\_hours
e number\_project. Il valore $1$ indica la correlazione con se stesso che infatti è massima.
Dal punto di vista della correlazione negativa, abbiamo debolmente correlati left con work\_accident e
satisfaction\_level con number\_projecy e time\_spend\_company. Abbiamo invece una correlazione negativa
moderata tra left e satisfaction\_level di valore $-0.39$.
% \section pairwise_correlations_and_eventual_elimination_od_redundant_variable (end)