\section{Data semantics} % (fold)
\label{sec:data_semantics}

In questo progetto viene analizzato il data set (simulato) \textbf{
  \href{https://www.kaggle.com/quentinvincenot/human-resources-analysis/data}{Human Resources Analytics}} che contiene
le informazioni sui dipendenti di un'azienda fittizia, suddivise in base alle variabili elencate nella Tabella \ref{col_tab}.

\begin{wraptable}{r}{8cm}
    \centering
    \begin{tabular}{| c | c |}
        \hline
        \textbf{Field} & \textbf{Type} \\ \hline
        satisfaction\_level & continuous \\ \hline
        last\_evaluation & continuous \\ \hline
        number\_project & discrete \\ \hline
        average\_montly\_hours & discrete \\ \hline
        time\_spend\_company & discrete \\ \hline
        work\_accident & categorical \\ \hline
        left & categorical \\ \hline
        promotion\_last\_5years & categorical \\ \hline
        sales & categorical \\ \hline
        salary & ordinal \\
        \hline
    \end{tabular}
    \caption{Variabili presenti nel Dataset e rispettivi tipi.}
    \label{col_tab}
\end{wraptable}

Con la prima variabile, \textit{Satisfaction Level}, viene
fornita una valutazione quantitativa del livello di soddisfazione di ciascun dipendente, in un range che va da un
valore minimo di $0$ ad un massimo di $1$. \textit{Last Evaluation} fornisce il tempo trascorso, in anni,
dall'ultima valutazione delle performace del dipendente. \textit{Number Project} riporta il numero di progetti
completati da ciascun dipendente durante il periodo di lavoro. \textit{Average Montly Hours} rappresenta la media
delle ore di lavoro in un mese. \textit{Time Spend Company} corrisponde al numero di anni trascorsi
dal dipendente all'interno dell'azienda. \textit{Work Accident} esprime con un $1$ il coinvolgimento di un
dipendente in un incidente sul lavoro, altrimenti viene impostato come $0$. \textit{Left} è utilizzata per tener
traccia dei dipendenti che hanno lasciato l'azienda, per i quali viene usato il valore $1$, mentre per quelli che sono
rimasti viene usato il valore $0$. \textit{Promotion last 5 Year} esprime con un $1$ se il dipendente
è stato promosso negli ultimi $5$ anni, altrimenti assume il valore $0$. \textit{Sales} definisce il dipartimento
nel quale il dipendente lavora, e \textit{Salary} esprime il livello (\textit{low}, \textit{medium},
\textit{high}) nel quale rientra il salario del dipendente.
Le descrizioni delle variabili sono state estrapolate dai metadati forniti assieme al Dataset sulla pagina di Kaggle
\footnote{\href{https://www.kaggle.com/}{https://www.kaggle.com/}} nella quale il Dataset è contenuto. La scelta
di catalogare le colonne \textit{Work Accident}, \textit{Left} e \textit{Promotion Last 5 Years} come
variabili di tipo categorico verrà opportunamente motivata nelle sezioni successive.

%% section data_semantics (end)
\section{Distribution of the variables and statistics} %% (fold)
\label{sec:distribution_of_the_variables_and_statistics}

In questo paragrafo vengono presentati i grafici relativi alla distribuzione dei valori assunti dalle variabili
descritte nella sezione precedente. Per dare una migliore interpretazione a questi abbiamo deciso di sostenere un'analisi accurata che contraddistingue i dipendenti che lavorano nell'azienda, rappresentati dal colore blu, e quelli che invece la hanno lasciata, rappresentati dal colore rosso.

Prima di tutto vogliamo studiare la distribuzione dei dipendenti rispetto alle variabili categoriche:
Divisione dipendenti totali per left , \includegraphics[scale=0.4]{images/sales.pdf}

Come si può constatare dal grafico sulla sinistra in un totale di $14999$ dipendenti analizzati: $3571$ di questi hanno lasciato l'azienda (quasi il $24\%$ del totale), mentre i restanti $11428$ dipendenti (quasi il $76\%$) sono rimasti nella azienda. La distribuzione mostrata sulla destra invece indica la distribuzione degli impiegati nei vari settori: (parte mancante) .

\includegraphics[scale=0.4]{images/work_accident.pdf}  \includegraphics[scale=0.4]{images/promotion_last_5_years.pdf}

Nel grafico a sinistra si studia il rapporto tra i dipendenti e la presenza o meno di un infortunio durante il periodo di lavoro all'interno dell'azienda e si è riscontrato che soltanto $169$ impiegati tra i $3571$ che se ne sono andati hanno avuto un incidente sul lavoro, mentre gli impiegati ancora all'interno dell'azienda ad aver subito un incidente sono $2000$ su $11428$.
    
Nel grafico a destra invece rapportiamo ciascun dipendente al fatto che questo sia stato promosso negli ultimi 5 anni oppure no, possiamo ricavare una informazione importante, la totalità degli impiegati che hanno lasciato l'azienda non ha
    avuto una promozione negli ultimi $5$ anni. Soltato $19$ impiegati su $11428$ di quelli rimasti sono stati promossi.
    
Una volta studiate le distribuzioni categoriche continuiamo l'analisi con gli altri attributi, per renderle più esplicative abbiamo affiancato a ciascun grafico la tabella che lo descrive in modo accurato e in modo che fosse autoesplicativo:

 Abbiamo rapportato in primis i dipendenti con il proprio salario, le percentuali che vengono indicate sul grafico non sono in base alla totalità dei dipendenti ma riguardano solo il tipo di dipendente definito dalla riga di appartenenza.
\includegraphics[scale=0.4]{images/salary.pdf} 

\begin{wraptable}{r}{8cm}
    \centering
    \begin{tabular}{| c | c | c | c |}
        \hline
        \textbf{Dipendente} & \textbf{Low} & \textbf{Medium} & \textbf{High}  \\ \hline
        InAzienda & 5144 (34.3\%) & & \\ \hline
        OutAzienda & (60\%) & &  \\ \hline
        \hline
    \end{tabular}
    \caption{Distribuzione salario per dipendente.}
    \label{col_tab}
\end{wraptable}
 

Proseguendo poi con il rapporto tra dipendenti ancora in azienda e non, e il numero delle ore di lavoro in media:
\includegraphics[scale=0.5]{images/average_montly_hours.pdf}

\begin{wraptable}{r}{8cm}
    \centering
    \begin{tabular}{| c | c | c | c |}
        \hline
        \textbf{Dipendente} & \textbf{media} & \textbf{dev. std.} & \textbf{min,max}  \\ \hline
        InAzienda & 199.06 & 45.68 & 96 , 287\\ \hline
        OutAzienda & 207.42 & 61.20 &  126 , 310\\ \hline
        \hline
    \end{tabular}
    \caption{Il tempo medio di ore di lavoro al mese per dipendente}
    \label{col_tab}
\end{wraptable}
      
 
 A questo punto è giusto analizzare il livello di soddisfazione e possiamo constatare dalla tabella che:
 \includegraphics[scale=0.5]{images/satisfaction_level.pdf}
\begin{wraptable}{r}{8cm}
    \centering
    \begin{tabular}{| c | c | c | c |}
        \hline
        \textbf{Dipendente} & \textbf{media} & \textbf{dev. std.} & \textbf{min,max}  \\ \hline
        InAzienda & 0.67 & 0.22 & 0.12 , 1.0\\ \hline
        OutAzienda & 0.44 & 0.26 & 0.09 , 0.92\\ \hline
        \hline
    \end{tabular}
    \caption{Livello di soddisfazione per il dipendente} 
    \label{col_tab}
\end{wraptable}

E la possiamo rapportare rispetto alla precedente valutazione del livello di soddisfazione (cambiare valori della tabella)
    \includegraphics[scale=0.5]{images/last_evaluation.pdf}
\begin{wraptable}{r}{8cm}
    \centering
    \begin{tabular}{| c | c | c | c |}
        \hline
        \textbf{Dipendente} & \textbf{media} & \textbf{dev. std.} & \textbf{min,max}  \\ \hline
        InAzienda & 0.67 & 0.22 & 0.12 , 1.0\\ \hline
        OutAzienda & 0.44 & 0.26 & 0.09 , 0.92\\ \hline
        \hline
    \end{tabular}
    \caption{Livello di soddisfazione per il dipendente} 
    \label{col_tab}
\end{wraptable}

% section distribution_of_the_variables_and_statistics (end)
\section{Assessing data quality} % (fold)
\label{sec:assessing_data_quality}
\lipsum
% section assessing_data_quality (end)
\section{Variable transformations} % (fold)
\label{sec:variable_transformations}
\lipsum
% section variable_transformations (end)
\section{Pairwise correlations and eventual elimination of redundant variable} % (fold)
\label{sec:pairwise_correlations_and_eventual_elimination_od_redundant_variable}
In questa sezione studieremo la correlazione ovvero la relazione lineare tra i vari attributi continuous 
o discreti.
\includegraphics[scale=0.5]{images/correlation_matrix.pdf}

% section pairwise_correlations_and_eventual_elimination_od_redundant_variable (end)
