\section{Data understanding} % (fold)
\label{sec:data_understanding}

% section data_understanding (end)
\section{Data semantics} % (fold)
\label{sec:data_semantics}
Il progetto viene svolto sul dataset (simulato) \textbf{
\href{https://www.kaggle.com/quentinvincenot/human-resources-analysis/data}{Human Resources Analytics}}. Questo
dataset contiene le informazioni sui dipendenti di un'azienda fittizia, suddivise in base ai seguenti campi:
\begin{description}
    \item[satisfaction\_level] - valutazione \textit{quantitativa} del livello di soddisfazione di ciascun
    dipendente, ha un valore compreso tra $0$ (minimo) e $1$ (massimo);
    \item[last\_evaluation] - tempo trascorso, in anni, dall'ultima valutazione delle performace del dipendente;
    \item[number\_project] - numero di progetti completati durante il periodo di lavoro;
    \item[average\_montly\_hours] - media delle ore di lavoro in un mese;
    \item[time\_spend\_company] - numero di anni trascorsi nell'azienda;
    \item[work\_accident] - esprime con un $1$ il coinvolgimento di un dipendente in un incidente sul lavoro,
    altrimenti viene impostato come $0$;
    \item[left] - se il dipendente ha lasciato l'azienda viene impostato come $1$, altrimenti come $0$;
    \item[promotion\_last\_5years] - se il dipendente è stato promosso negli ultimi $5$ anni viene impostato come
    $1$, altrimenti come $0$;
    \item[sales] - definisce il dipartimento nel quale il dipendente lavora;
    \item[salary] - esprime il livello (\textit{low}, \textit{medium}, \textit{high}), di salario nel quale
    rientra il dipendente;
\end{description}
% section data_semantics (end)
\section{Distribution of the variables and statistics} % (fold)
\label{sec:distribution_of_the_variables_and_statistics}

% section distribution_of_the_variables_and_statistics (end)
\section{Assessing data quality} % (fold)
\label{sec:assessing_data_quality}

% section assessing_data_quality (end)
\section{Variable transformations} % (fold)
\label{sec:variable_transformations}

% section variable_transformations (end)
\section{Pairwise correlations and eventual elimination od redundant variable} % (fold)
\label{sec:pairwise_correlations_and_eventual_elimination_od_redundant_variable}

% section pairwise_correlations_and_eventual_elimination_od_redundant_variable (end)
