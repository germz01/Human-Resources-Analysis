\section{Data semantics} % (fold)
\label{sec:data_semantics}

In questo progetto viene analizzato il data set (simulato) \textbf{
  \href{https://www.kaggle.com/quentinvincenot/human-resources-analysis/data}{Human Resources Analytics}} che contiene
le informazioni sui dipendenti di un'azienda fittizia, suddivise in base alle variabili elencate nella Tabella \ref{col_tab}.

\begin{wraptable}{r}{8cm}
    \centering
    \begin{tabular}{| c | c |}
        \hline
        \textbf{Field} & \textbf{Type} \\ \hline
        satisfaction\_level & continuous \\ \hline
        last\_evaluation & continuous \\ \hline
        number\_project & discrete \\ \hline
        average\_montly\_hours & discrete \\ \hline
        time\_spend\_company & discrete \\ \hline
        work\_accident & categorical \\ \hline
        left & categorical \\ \hline
        promotion\_last\_5years & categorical \\ \hline
        sales & categorical \\ \hline
        salary & ordinal \\
        \hline
    \end{tabular}
    \caption{Variabili presenti nel Dataset e rispettivi tipi.}
    \label{col_tab}
\end{wraptable}

Con la prima variabile, \textit{Satisfaction Level}, viene
fornita una valutazione quantitativa del livello di soddisfazione di ciascun dipendente, in un range che va da un
valore minimo di $0$ ad un massimo di $1$. \textit{Last Evaluation} fornisce il tempo trascorso, in anni,
dall'ultima valutazione delle performace del dipendente. \textit{Number Project} riporta il numero di progetti
completati da ciascun dipendente durante il periodo di lavoro. \textit{Average Montly Hours} rappresenta la media
delle ore di lavoro in un mese. \textit{Time Spend Company} corrisponde al numero di anni trascorsi
dal dipendente all'interno dell'azienda. \textit{Work Accident} esprime con un $1$ il coinvolgimento di un
dipendente in un incidente sul lavoro, altrimenti viene impostato come $0$. \textit{Left} è utilizzata per tener
traccia dei dipendenti che hanno lasciato l'azienda, per i quali viene usato il valore $1$, mentre per quelli che sono
rimasti viene usato il valore $0$. \textit{Promotion last 5 Year} esprime con un $1$ se il dipendente
è stato promosso negli ultimi $5$ anni, altrimenti assume il valore $0$. \textit{Sales} definisce il dipartimento
nel quale il dipendente lavora, e \textit{Salary} esprime il livello (\textit{low}, \textit{medium},
\textit{high}) nel quale rientra il salario del dipendente.
Le descrizioni delle variabili sono state estrapolate dai metadati forniti assieme al Dataset sulla pagina di Kaggle
\footnote{\href{https://www.kaggle.com/}{https://www.kaggle.com/}} nella quale il Dataset è contenuto. La scelta
di catalogare le colonne \textit{Work Accident}, \textit{Left} e \textit{Promotion Last 5 Years} come
variabili di tipo categorico verrà opportunamente motivata nelle sezioni successive.

%% section data_semantics (end)
\section{Distribution of the variables and statistics} %% (fold)
\label{sec:distribution_of_the_variables_and_statistics}

In questo paragrafo vengono presentati i grafici relativi alla distribuzione dei valori assunti dalle variabili
descritte nella sezione precedente. Per dare una migliore interpretazione a questi grafici abbiamo deciso di sostenere un'analisi accurata che contraddistingue i dipendenti che lavorano nell'azienda, rappresentati dal colore blu, e quelli che invece la hanno lasciata, rappresentati dal colore rosso.

Prima di tutto vogliamo mostrare le caratteristiche fondamentali alla nostra ricerca:
Barplot dipendenti totali (suddivisione left), barplot sales per employee

Dati questi rapporti fondamentali abbiamo iniziato la nostra analisi:

\begin{figure}[H]
    \captionsetup[subfigure]{font=scriptsize,labelfont=scriptsize}
    \centering
    \subcaptionbox{\includegraphics[scale=0.4]{images/salary.pdf}}
 \end{figure}
 Abbiamo iniziato a rapportare i dipendenti con il proprio salario, e si è evinto che       in un totale di 14999 dipendenti analizzati: 3571 di questi hanno lasciato l'azienda (quasi il 24\% del totale (23.82\%)), di cui il 60\% di questi aveva uno stipendio low, i restanti 11428 dipendenti (quasi il 76\%) sono rimasti nella azienda di cui 5144 (il 34.3\%) ha un salario low, (mancano dettagli per high e medium).
     
Proseguendo poi con il rapporto tra dipendenti ancora in azienda e non e il numero delle ore di lavoro in media: scopriamo che Per gli impiegati che hanno lasciato l'azienda, la \textbf{media} delle ore di lavoro mensili è
    $207.42$, la \textbf{deviazione standard} è $61.20$, i valori \textbf{minimo} e \textbf{massimo} sono,
    rispettivamente, $126$ e $310$. Per quanto riguarda gli impiegati rimasti abbiamo invece una \textbf{media}
    pari a $199.06$, una \textbf{deviazione standard} pari a $45.68$ e un valore \textbf{minimo} e
    \textbf{massimo} pari a $96$ e $287$.\label{average_montly_hours}    
 \begin{figure}[H]
    \captionsetup[subfigure]{font=scriptsize,labelfont=scriptsize}
    \centering    
    \subcaptionbox{}
    {\includegraphics[scale=0.5]{images/average_montly_hours.pdf}}
\end{figure}

Se rapportiamo i dipendenti ai valori che indicano se il dipendente è stato promosso negli ultimi 5 anni o no, possiamo ricavare una informazione importante, la totalità degli impiegati che hanno lasciato l'azienda non ha
    avuto una promozione negli ultimi $5$ anni. Soltato $19$ impiegati su $3571$ sono stati promossi.
\begin{figure}[H]
    \captionsetup[subfigure]{font=scriptsize,labelfont=scriptsize}
    \centering
    \subcaptionbox{}
    {\includegraphics[scale=0.4]{images/promotion_last_5_years.pdf}}
\end{figure}
Soltanto $169$ impiegati tra i $3571$ che se ne sono andati ha avuto un incidente sul lavoro,
    mentre gli impiegati ancora all'interno dell'azienda ad aver subito un incidente sono $2000$ su $11428$
    \label{work_accident}
\begin{figure}[H]
    \captionsetup[subfigure]{font=scriptsize,labelfont=scriptsize}
    \centering
    \subcaptionbox{\includegraphics[scale=0.4]{images/work_accident.pdf}}
 \end{figure}
 
 A questo punto è giusto analizzare il livello di soddisfazione e possiamo constatare che per gli impiegati che hanno lasciato l'azienda, la \textbf{media} del livello di soddisfazione è $0.44$, la \textbf{deviazione standard} è $0.26$, i valori \textbf{minimo} e \textbf{massimo} sono, rispettivamente, $0.09$ e $0.92$. Per quanto riguarda gli impiegati rimasti abbiamo invece una \textbf{media} pari a $0.67$, una \textbf{deviazione standard} pari a $0.22$ e un valore \textbf{minimo} e \textbf{massimo}
pari a $0.12$ e $1.0$.\label{satisfaction_level}
  \begin{figure}[H]
    \captionsetup[subfigure]{font=scriptsize,labelfont=scriptsize}
    \centering
    \subcaptionbox{\includegraphics[scale=0.5]{images/satisfaction_level.pdf}}
     \end{figure}
E la possiamo rapportare rispetto alla precedente valutazione del livello di soddisfazione
 \begin{figure}[H]
    \captionsetup[subfigure]{font=scriptsize,labelfont=scriptsize}
    \centering 
    \subcaptionbox{\label{last_evaluation}}{\includegraphics[scale=0.5]{images/last_evaluation.pdf}}
\end{figure}
% section distribution_of_the_variables_and_statistics (end)
\section{Assessing data quality} % (fold)
\label{sec:assessing_data_quality}
\lipsum
% section assessing_data_quality (end)
\section{Variable transformations} % (fold)
\label{sec:variable_transformations}
\lipsum
% section variable_transformations (end)
\section{Pairwise correlations and eventual elimination of redundant variable} % (fold)
\label{sec:pairwise_correlations_and_eventual_elimination_od_redundant_variable}
\lipsum
% section pairwise_correlations_and_eventual_elimination_od_redundant_variable (end)
