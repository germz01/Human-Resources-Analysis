\section{Clustering Analysis by K-means} % (fold)
\label{sec:clustering_analysis_by_k_means}
\subsection{Choice of attributes and distance function} % (fold)
\label{sub:choice_of_attributes_and_distance_function}
Le variabili sulle quali abbiamo deciso di applicare la Cluster Analysis tramite K-means sono le due
variabili di tipo continous presenti nel Dataset, ossia \textit{Satisfaction Level} e \textit{Latest Evaluation}. Le
variabili di tipo categorico sono state scartate al momento della scelta dato che la natura stessa dell'algoritmo
prevede il suo utilizzo su variabili di tipo numerico. \\ \\ Approfondire su variabili discrete \\ \\
Nell'implementazione dell'algoritmo da noi utilizzata è stato deciso di applicare la distanza euclidea come distance
function.
% subsection choice_of_attributes_and_distance_function (end)
\subsection{Identification of the best value of k} % (fold)
\label{sub:identification_of_the_best_value_of_k}
\begin{figure}[b]
    \centering
    \subcaptionbox{\label{fig:sse_left}}{\includegraphics[scale=0.5]{images/kmeans/SSE_left.pdf}}
    \subcaptionbox{\label{fig:sse_stayed}}{\includegraphics[scale=0.5]{images/kmeans/SSE_stayed.pdf}}
    \caption{Prova}
\end{figure}
Al fine di identificare il miglior numero $k$ di clusters da utilizzare, abbiamo tenuto conto dell'Error Sum of
Squares (SSE) per ogni iterazione dell'algoritmo, svolta a partire da un valore iniziale di $k$ pari a $2$ fino
ad un valore massimo di $50$. Rappresentato in Figura \ref{fig:sse_left} troviamo l'andamento dell'SSE per i
cluster relativi ai dipendenti che hanno lasciato l'azienda. Possiamo notare il valore ottimale di $k$ pari a $3$,
che è la posizione sull'asse dei cluster dove la curva inizia il suo percorso discendente. In figura
\ref{fig:sse_stayed} possiamo invece vedere la stessa cosa, ma per i dipendenti che sono rimasti all'interno
dell'azienda. In questo caso notiamo il valore ottimale di $k$ pari a $5$.
% subsection identification_of_the_best_value_of_k (end)
\subsection{Characterization of the obtained clusters by using both analysis of the k centroids and comparison of the distribution of variables within the clusters and that in the whole dataset} % (fold)
\label{sub:characterization_of_the_obtained_clusters_by_using_both_analysis_of_the_k_centroids_and_comparison_of_the_distribution_of_variables_within_the_clusters_and_that_in_the_whole_dataset}

% subsection characterization_of_the_obtained_clusters_by_using_both_analysis_of_the_k_centroids_and_comparison_of_the_distribution_of_variables_within_the_clusters_and_that_in_the_whole_dataset (end)
% section clustering_analysis_by_k_means (end)
