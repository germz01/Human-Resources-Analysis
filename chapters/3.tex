Per definire le association rules prima si sono seguite le seguenti operazioni:
\begin{enumerate}
\item Abbiamo trasformato le variabili da stringhe a numeriche, per quanto riguarda gli attributi $salary$ e $department$.
\item Successivamente abbiamo raggruppato le variabili $last_evaluation$ e $satisfaction_level$ e $average_monlty_hours$, con un intervallo di $0.10$ ciascuna. 
\item Infine per rendere unici i risultati numerici, è stata aggiunta una stringa subito successiva al numero in modo tale che non fosse ambiguo ma immediato quel valore a che attributo si riferisca.
\end{enumerate}


\section{Frequent patterns extraction with different values of support and different types }
Una volta eseguiti i passi detti in precedenza si è eseguita l'estrazione dei pattern frequenti tramite la metodologia Apriori. 


\section{Discussion of the most interesting frequent patterns}
I frequent pattern ritenuti più interessanti sono i seguenti:
-
-
-

Da questi scaturisce che ... 


\section{Association rules extraction with different values of confidence }




\section{Discussion of the most interesting rules }
Abbiamo inoltre rilevato delle rules che nonostante non siano frequenti sono comunque interessanti per l'analisi che stiamo portando avanti. Queste sono:
-
-
Da queste possiamo ricavare che...



\section{Use the most meaningful rules to replace missing values and evaluate the accuracy}
Nel nostro dataset non sono presenti missing values quindi non è stata necessaria la valutazione delle più significative rules e la valutazione dell'accuratezza per rimpiazzare questi.


\section{Use the most meaningful rules to predict if an employee will leave prematurely or not and evaluate the accuracy }
Date le varie association rules trovate dalle varie prove queste sono quelle più significative per predirre se un impiegato lascerà prematuramente l'azienda oppure no:

- AR ... accuratezza trovata: 
- AR1
- AR2

Da queste possiamo scaturire che un impiegato lasci il posto di lavoro prematuramente quando è nelle seguenti condizioni:

Invece rimarrà quando avrà una condizione del tipo: 







