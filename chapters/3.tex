Per definire le association rules prima si sono seguite le seguenti operazioni:
\begin{itemize}
\item Abbiamo trasformato le variabili da stringhe a numeriche, per quanto riguarda gli attributi \textit{Salary}
e \textit{Department}.
\item Abbiamo raggruppato le variabili \textit{Last Evaluation}, \textit{Satisfaction Level} e
\textit{Average Montly Hours}, utilizzando il numero di bins ricavato dall'applicazione della formula di Sturges
$\lceil \log{n}_2 \rceil + 1$, dove $n$ in questo caso è $14999$, il numero di data objects.
\item Per rendere unici i risultati numerici, è stata aggiunta una stringa subito successiva al valore numerico
in modo da non renderlo ambiguo e sopratutto in modo da poter capire univocamente a che attributo si riferisca.
\end{itemize}
\section{Frequent patterns extraction with different values of support and different types} % (fold)
\label{sec:frequent_patterns_extraction_with_different_values_of_support_and_different_types}
Dopo aver eseguito i passi preliminari descritti precedentemente abbiamo svolto l'analisi inerente ai
\textit{frequent patterns} attraverso l'applicazione dell'algoritmo \textit{Apriori}. Per ogni iterazione
dell'algoritmo, abbiamo considerato, indipendentemente dal \textit{support}, gli itemsets con $2$ o più items al
loro interno. Inoltre, al fine di avere una panoramica più completa, abbiamo svolto l'analisi per i
\textit{frequent itemsets}, per i \textit{closed frequent itemsets} e per i \textit{maximal frequent itemsets}.
Abbiamo quindi cominciato l'analisi con un support pari a $20$, ossia prendendo in considerazione soltanto gli
itemsets presenti in almeno il $20\%$ della transazioni. Successivamente abbiamo utilizzato un support pari a
$30$. Le quantità di frequent itemsets scoperte al variare dei paramentri sono riportate in Tabella
\ref{tab:freq_items}.
\begin{table}[H]
    \centering
    \resizebox{0.8\textwidth}{!}{
        \begin{tabular}{| c | c | c | c |}
            \hline
            \textbf{Support Threshold} & \textbf{Frequent Itemsets} & \textbf{Closed Frequent Itemsets} &
            \textbf{Maximal Frequent Itemsets} \\ \hline
            $20$ & $52$ & $52$ & $10$ \\ \hline
            $30$ & $19$ & $19$ & $7$ \\ \hline
        \end{tabular}
    }
    \caption{Quantità di frequent itemsets trovati per ogni tipologia e support utilizzati durante l'analisi.}
    \label{tab:freq_items}
\end{table}
Come era lecito aspettarsi, esiste un rapporto di proporzionalità inversa tra la soglia di support e il numero di
frequent itemsets scoperti.
% section frequent_patterns_extraction_with_different_values_of_support_and_different_types (end)
\section{Discussion of the most interesting frequent patterns} % (fold)
\label{sec:discussion_of_the_most_interesting_frequent_patterns}
Passiamo adesso alla descrizione dei frequent items più interessanti che sono stati scoperti durante l'analisi. In
Tabella \ref{tab:interesting:freq_items} vengono riportati gli itemsets più interessanti dal punto di vista del
supporto pari a $20$ scoperti durante l'analisi.
\begin{table}[H]
    \centering
    \resizebox{\textwidth}{!}{
        \begin{tabular}{| c | c | c | c | c | c |}
            \hline
            \textbf{Frequent Itemsets ($ST = 20$)} & \textbf{Support} &
            \textbf{Closed Frequent Itemsets ($ST = 20$)} & \textbf{Support} &
            \textbf{Maximal Frequent Itemsets ($ST = 20$)} & \textbf{Support}\\
            \hline
            (N\_WA, \ N\_P)  & $0.84$ & (N\_WA, \ N\_P)  & $0.84$ & (0\_S, \ N\_L, \ N\_WA, \ N\_P) & $0.28$\\
            \hline
            (N\_L, \ N\_P) & $0.74$ & (N\_L, \ N\_P) & $0.74$ & (1\_S, \ N\_L, \ N\_WA, \ N\_P) & $0.28$ \\
            \hline
            (N\_L, \ N\_WA) & $0.63$ & (N\_L, \ N\_WA) & $0.63$ & (3\_T, N\_L, N\_WA, N\_P) & $0.26$ \\
            \hline
            (N\_L, \ N\_WA, \ N\_P) & $0.61$ & (N\_L, \ N\_WA, \ N\_P) & $0.61$ & (0\_D, \ N\_WA, \ N\_P) &
            $0.23$ \\ \hline
            (0\_S, \ N\_P) & $0.48$ & (0\_S, \ N\_P) & $0.48$ & (Y\_L, \ N\_WA, \ N\_P) & $0.23$ \\
            \hline
        \end{tabular}
    }
    \caption{Frequent itemsets con supporto maggiore scoperti durante l'analisi utilizzando un supporto pari a
    $20$.}
    \label{tab:interesting:freq_items}
\end{table}
Possiamo vedere che i frequent itemsets e i closed frequent itemsets sono gli stessi, quindi verranno descritti
una volta per entrambi. Possiamo notare come il $2$-itemset
% section discussion_of_the_most_interesting_frequent_patterns (end)
\section{Association rules extraction with different values of confidence }




\section{Discussion of the most interesting rules }
Abbiamo inoltre rilevato delle rules che nonostante non siano frequenti sono comunque interessanti per l'analisi che stiamo portando avanti. Queste sono:
-
-
Da queste possiamo ricavare che...



\section{Use the most meaningful rules to replace missing values and evaluate the accuracy}
Nel nostro dataset non sono presenti missing values quindi non è stata necessaria la valutazione delle più significative rules e la valutazione dell'accuratezza per rimpiazzare questi.


\section{Use the most meaningful rules to predict if an employee will leave prematurely or not and evaluate the accuracy }
Date le varie association rules trovate dalle varie prove queste sono quelle più significative per predirre se un impiegato lascerà prematuramente l'azienda oppure no:

- AR ... accuratezza trovata:
- AR1
- AR2

Da queste possiamo scaturire che un impiegato lasci il posto di lavoro prematuramente quando è nelle seguenti condizioni:

Invece rimarrà quando avrà una condizione del tipo:







