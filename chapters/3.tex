Per definire le association rules prima si sono seguite le seguenti operazioni:
\begin{itemize}
\item Abbiamo trasformato le variabili da stringhe a numeriche, per quanto riguarda gli attributi \textit{Salary}
e \textit{Department}.
\item Abbiamo raggruppato le variabili \textit{Last Evaluation}, \textit{Satisfaction Level} e
\textit{Average Montly Hours}, usando $3$ bins sia per la prima che per la seconda variabile, usando intervalli
specifici, e $2$ bins per la terza variabile, applicando anche in questo caso una suddivisione ad hoc.
\item Per rendere unici i risultati numerici, è stata aggiunta una stringa subito successiva al valore numerico
in modo da non renderlo ambiguo e sopratutto in modo da poter capire univocamente a che attributo si riferisca.
\end{itemize}
\todo[inline]{Aggiungere la divisione in intervalli scelta e legenda con le abbreviazioni, oppure sotto non abbreviare}
\section{Frequent patterns extraction with different values of support and different types} % (fold)
\label{sec:frequent_patterns_extraction_with_different_values_of_support_and_different_types}
Dopo aver eseguito i passi preliminari descritti precedentemente abbiamo svolto l'analisi inerente ai
\textit{frequent patterns} attraverso l'applicazione dell'algoritmo \textit{Apriori}. Per ogni iterazione
dell'algoritmo, abbiamo considerato, indipendentemente dal \textit{support}, gli itemsets con $2$ o più items al
loro interno. Inoltre, al fine di avere una panoramica più completa, abbiamo svolto l'analisi per i
\textit{frequent itemsets}, per i \textit{closed frequent itemsets} e per i \textit{maximal frequent itemsets}.
Abbiamo quindi cominciato l'analisi con un support pari a $20$, ossia prendendo in considerazione soltanto gli
itemsets presenti in almeno il $20\%$ della transazioni. Successivamente abbiamo utilizzato un support pari a
$30$. Le quantità di frequent itemsets scoperte al variare dei paramentri sono riportate in Tabella
\ref{tab:freq_items}.
\begin{table}[H]
    \centering
    \resizebox{0.8\textwidth}{!}{
        \begin{tabular}{| c | c | c | c |}
            \hline
            \textbf{Support Threshold} & \textbf{Frequent Itemsets} & \textbf{Closed Frequent Itemsets} &
            \textbf{Maximal Frequent Itemsets} \\ \hline
            $20$ & $137$ & $130$ & $30$ \\ \hline
            $30$ & $46$ & $45$ & $11$ \\ \hline
        \end{tabular}
    }
    \caption{Quantità di frequent itemsets trovati per ogni tipologia e support utilizzati durante l'analisi.}
    \label{tab:freq_items}
\end{table}
Come era lecito aspettarsi, esiste un rapporto di proporzionalità inversa tra la soglia di support e il numero di
frequent itemsets scoperti.
% section frequent_patterns_extraction_with_different_values_of_support_and_different_types (end)
\section{Discussion of the most interesting frequent patterns} % (fold)
\label{sec:discussion_of_the_most_interesting_frequent_patterns}
Passiamo adesso alla descrizione dei frequent items più interessanti che sono stati scoperti durante l'analisi. In
Tabella \ref{tab:interesting:freq_items} vengono riportati gli itemsets più interessanti dal punto di vista del
supporto pari a $20$ scoperti durante l'analisi.
\begin{table}[H]
    \centering
    \resizebox{\textwidth}{!}{
        \begin{tabular}{| c | c | c | c | c | c |}
            \hline
            \textbf{Frequent Itemsets ($ST = 20$)} & \textbf{Support} &
            \textbf{Closed Frequent Itemsets ($ST = 20$)} & \textbf{Support} &
            \textbf{Maximal Frequent Itemsets ($ST = 20$)} & \textbf{Support}\\
            \hline
            (N\_WA, \ N\_P)  & $0.84$ & (N\_WA, \ N\_P)  & $0.84$ & (standard\_H, N\_L, N\_WA, N\_P) & $0.31$ \\
            \hline
            (N\_L, \ N\_P) & $0.74$ & (N\_L, \ N\_P) & $0.74$ & (intensive\_H, N\_L, N\_WA, N\_P) & $0.30$ \\
            \hline
            (N\_L, \ N\_WA) & $0.63$ & (N\_L, \ N\_WA) & $0.63$ & (0\_S, \ N\_L, \ N\_WA, \ N\_P) & $0.28$ \\
            \hline
            (N\_L, \ N\_WA, \ N\_P) & $0.61$ & (N\_L, \ N\_WA, \ N\_P) & $0.61$ & (1\_S, N\_L, N\_WA, N\_P) &
            $0.28$ \\ \hline
        \end{tabular}
    }
    \caption{Frequent itemsets con supporto maggiore scoperti durante l'analisi utilizzando un supporto pari a
    $20$. Con N\_WA intendiamo l'item relativo all'assenza di incidenti sul lavoro, con N\_P l'item relativo alla
    mancanza di promozioni, con N\_L l'item relativo ai dipendenti ancora in azienda, con 0\_S l'item relativo ai
    dipendenti con salario minimo, con 1\_S l'item relativo ai dipendenti con salario medio, con intensive\_H
    intendiamo i dipendenti con un quantitativo di ore mensili compreso tra $200$ e $300$ e con standard\_H
    intendiamo i dipendenti con un quantitativo di ore mensili inferiore a $200$.}
    \label{tab:interesting:freq_items}
\end{table}
Descriviamo per primi i frequent itemsets e i closed frequent itemsets, visto che sono identici. Possiamo notare
come la situazione presentata proponga in maggioranza impiegati i quali non hanno subito incidenti sul lavoro,
che non sono stati promossi e che non hanno lasciato l'azienda. Per quanto riguarda i maximal frequent itemsets
troviamo che gli impiegati con carichi di lavoro sia standard che elevati, che non hanno lasciato l'azienda, non
hanno avuto incidenti sul lavoro e che non sono stati promossi negli ultimi $5$ anni sono i più diffusi, seguiti
dagli impiegati di salario minimo e medio, non promossi e i quali non hanno avuto incidenti sul lavoro.
Portando la soglia del support a $30$, gli itemsets più diffusi sono gli stessi che sono  stati descritti per la
soglia pari a $20$, evitiamo quindi di descriverli.
% section discussion_of_the_most_interesting_frequent_patterns (end)
\section{Association rules extraction with different values of confidence }

Per estrarre le associazioni interessanti del dataset sono stati utilizzati due approcci principali per la scelta dei parametri dell'algoritmo \textit{apriori}, in modo da ottenere:
\begin{itemize}
\item \textit{Regole "generali''}, ovvero associazioni interessanti valide per un numero ampio di impiegati, ottenute fissando un alto supporto minimo, pari a $MinSupp = 20\%$, nella ricerca degli itemset frequenti.
\item \textit{Regole "specifiche''}: il supporto minimo è stato fissato considerando che uno degli obiettivi principali delle analisi contenute in questo report è capire il perchè una parte consistente dei dipendenti ha lasciato l'azienda. 
La percentuale di dipendenti che hanno lasciato l'azienda corrisponde al $24\%$ del totale, assumendo come significativa una regola che riguardi almeno il $20\%$ dei dipendenti che hanno lasciato l'azienda, risulta un supporto minimo pari a circa $MinSupp=5\%$.
\end{itemize}

L'analisi è stato eseguita per entrambi i valori di $MinSupp$ indicati, variando la confidenza minima \textit{MinConf} per valori compresi tra $50-100\%$. In Fig. sono riportate il numero di regole ottenute, in funzione della confidenza.
Le regole cercate hanno un solo item come parte conseguente, per facilitare l'analisi e ridurre il numero di regole in partenza.
Per ciascuna regola estratta sono stati calcolati gli indici di \textit{confidence} e \textit{lift} in modo da valutare l'interesse oggettivo delle associazioni trovate.

\begin{figure}[htbp]
  \centering
  \includegraphics[width=0.9\textwidth]{../images/rules/andamentonumero.pdf}
  \caption{Numero di regole estratte al variare di \textit{MinConf}, per differenti valori di \textit{MinSupp}.}
  \label{fig:rules-number}
\end{figure}



\subsection{Regole generali}
Le regole generali sono state cercate fissando a priori \textit{Minsupp=20\%} e considerando un valore di confidenza minimo \textit{MinConf=50\%} ,
scelto osservando l'andamento in Fig. \ref{fig:rules-number}, sufficiente per ottenere un numero trattabile di regole. 

Per un valore di \textit{MinConf=50\%} sono state ottenute 268 regole  rappresentate in Fig. \ref{fig:rules_general}, con i rispettivi valori di Lift, Confidence e Support. Si osserva che la maggior parte delle regole ha un Lift basso, minore di 1.5.
Le regole sono state rappresentate raffigurando il valore di $Conf$ in funzione di $Lift$. Questo permette di identificare visivamente in modo immediato le regole con la stessa parte conseguente, poichè in tal caso i due indici sono proporzionali con coefficiente pari al supporto della parte conseguente.
Le regole che possono essere considerate oggettivamente interessanti, con $Lift> 1.5$, visibili nella parte destra del grafico sono riportate in Tab \ref{tab:rules_general} e mostrano un legame positivo tra un alto livello di valutazione (\textit{very good, Last Evaluation}) ed un alto grado di soddisfazione degli impiegati, a cui si accompagnano nella parte antecedente la mancanza di incidenti sul lavoro (\textit{N WA}) ed il fatto di essere rimasti in azienda.

Tutti gli impiegati che hanno una \textit{'Last evaluation: very good'} hanno anche un alto tasso di soddisfazione ('high\_SL'), mentre il viceversa accade nel $68\%$ dei casi. Queste regole non sono particolarmente interessanti dal punto di vista dell'analisi, poichè abbastanza ovvie, e non aggiungono molto alle analisi statistiche eseguite.

  
\begin{minipage}[c]{0.55\textwidth}
\begin{figure}[H]
  \centering
  \includegraphics[width=1\textwidth]{../images/rules/rules_general.pdf}
  \caption{Grafico,mettere LEGENDA PER COLORE E SIZE (Support)}
  \label{fig:rules_general}
\end{figure}
\end{minipage}
\begin{minipage}[c]{0.45\textwidth}
\begin{table}[H]
  \centering
  \scalebox{0.7}{
\input{data/rules/rules_supp20.txt}  
}
\caption{Association rules per $MinSupp=20\%$ e $Lift> 1.5$}
\label{tab:rules_general}
\end{table}
\end{minipage}


\subsection{Regole specifiche}

Usando $MinSupp=5\%$ naturalmente il numero di associazioni trovate cresce notevolmente, raggiungendo in questo caso le oltre 6000 regole.
Le regole sono state dunque filtrate ulteriormente in modo da avere un interesse oggettivo considerando $Lift>1$ e $Conf>0.9$, ed inoltre $Supp < 20\%$, poichè per supporto maggiore sono state analizzate in precedenza. Le circa $2000$ regole ottenute sono rappresentate in Fig. \ref{fig:rules_specific}, si può osservare che le regole con Lift più alto hanno come parte conseguente, partendo dalla destra del grafico, \textit{Satisfaction Level: low , Number of Project: 2, Left : Y, Last Evaluation: Insufficient}.

\begin{figure}[H]
  \centering
  \includegraphics[width=0.5\textwidth]{../images/rules/rules_specific.pdf}
  \caption{Grafico,mettere LEGENDA PER COLORE E SIZE (Support)}
  \label{fig:rules_specific}
\end{figure}


\subsubsection{Regole per impiegati che hanno lasciato }
Tra le regole specifiche sono state estratte solo quelle con parte conseguente uguale a \textit{Left:YES}.
Per filtrarle ulteriormente è stato usato un ulteriore criterio, che riportiamo insieme a tutti gli step successivi che hanno portato alla
selezione delle regole più interessanti, riguardanti gli impiegati che hanno lasciato l'azienda, riportate in Fig. \ref{fig:rules_left}:
%Criterio pruning regole: confidenza maggiore di 0.9, a parità di supporto (all'1/1000) si considera la regola più corta contenente le altre

\begin{enumerate}
\item $MinSupp = 5\%$ e $MinConf = 50\%$ nell'algoritmo \textit{apriori}, ottenendo $N_{rules}=264$ per la parte conseguente $Left: Yes$ ; 
\item rimozione delle regole contenenti $Work Accident:No$ , poichè non è semanticamente sensata l'implicazione di $Left: Yes$, ottenendo $N_{rules}=136$; 
\item filtro a posteriori con $Lift>1$ e  $Conf>0.85$, ottenendo $N_{rules}=62$

\item Per le regole con itemsets delle parti antecedenti sottoinsiemi l'uno dell'altro e \textit{con supporto uguale} (confronto all'1/1000), sono state considerate le regole con parte antecedente più lunga,
  Ad esempio per due regole del tipo $r_1:\, A \rightarrow Y$, $r2:\,AB \rightarrow Y$, con $A,B,Y$ differenti item, e $Supp_1= Supp_2$ si seleziona solo la regola $r_2$. 
\end{enumerate}
%L'ultimo criterio permette di non ripetere regole che hanno la stessa base di applicazione, ovvero lo stesso supporto.
Le 12 regole ottenute con il procedimento spiegato sono riportate in Fig. \ref{fig:rules_left} insieme ai rispettivi valori di confidence e support.
E' inoltre riportato il supporto relativo alle persone che hanno lasciato, $Supp\, Left$.
Le regole sono rappresentate in verticale, con i valori $10$ e $0$ che indicano la presenza o meno di un item nella parte antecedente della regola.

\begin{figure}[htbp]
  \centering
  \includegraphics[width=0.99\textwidth]{../images/rules/left_heatmap.pdf}
  \caption{Regole associative più interessanti con parte conseguente gli impiegati che hanno lasciato l'azienda. I valori 10-0 corrispondono alla presenza o meno di un item nella parte antecedente della regola}
  \label{fig:rules_left}
\end{figure}

Il supporto massimo risulta pari al $10\%$, corrispondente ad una percentuale di circa il $40\%$ tra gli impiegati che hanno lasciato.
Si osserva dunque una correlazione positiva tra gli item riportati in Fig. \ref{fig:rules_left} e il fatto di aver lasciato l'azienda.

In generale le regole mostrano che le persone che lasciano l'azienda sono poco stimolate, risulta che circa il $40\%$ degli impiegati che hanno lasciato l'azienda lavora a solo due progetti, con un impegno di ore mensili basso, una soddisfazione media, ed una presenza in azienda da 3 anni. Altri aspetti rilevanti sono la mancanza di una promozione, una valutazione insufficiente ed un salario basso.


\subsubsection{Altre regole interessanti}

Dalle regole specifiche estratte ne sono state ricavate alcune interessanti, per alto valore di $Conf$ e $Lift$, riportate in Tab.
\ref{tab:altreregole}

\begin{table}[htbp]
  \centering
\input{"data/rules/altreregole.txt"}
\caption{Altre regole interessanti}
\label{tab:altreregole}
\end{table}

\clearpage
\newpage


\section{Use the most meaningful rules to predict if an employee will leave prematurely or not and evaluate the accuracy }
Date le varie association rules trovate dalle varie prove queste sono quelle più significative per predirre se un impiegato lascerà prematuramente l'azienda oppure no:

- AR ... accuratezza trovata:
- AR1
- AR2

Da queste possiamo scaturire che un impiegato lasci il posto di lavoro prematuramente quando è nelle seguenti condizioni:

Invece rimarrà quando avrà una condizione del tipo:







