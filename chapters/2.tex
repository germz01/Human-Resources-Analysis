\section{Clustering Analysis by K-means} % (fold)
\label{sec:clustering_analysis_by_k_means}
\subsection{Choice of attributes and distance function} % (fold)
\label{sub:choice_of_attributes_and_distance_function}
Le variabili che abbiamo deciso di utilizzare al fine di realizzare la Cluster Analysis tramite K-means sono le due
variabili di tipo continous presenti nel Dataset, ossia \textit{Satisfaction Level} e \textit{Latest Evaluation}. Le
variabili di tipo categorico sono state scartate al momento della scelta dato che la natura stessa dell'algoritmo
prevede il suo utilizzo su variabili di tipo numerico. \\ \\ Approfondire su variabili discrete \\ \\
% subsection choice_of_attributes_and_distance_function (end)
\subsection{Identification of the best value of k} % (fold)
\label{sub:identification_of_the_best_value_of_k}

% subsection identification_of_the_best_value_of_k (end)
\subsection{Characterization of the obtained clusters by using both analysis of the k centroids and comparison of the distribution of variables within the clusters and that in the whole dataset} % (fold)
\label{sub:characterization_of_the_obtained_clusters_by_using_both_analysis_of_the_k_centroids_and_comparison_of_the_distribution_of_variables_within_the_clusters_and_that_in_the_whole_dataset}

% subsection characterization_of_the_obtained_clusters_by_using_both_analysis_of_the_k_centroids_and_comparison_of_the_distribution_of_variables_within_the_clusters_and_that_in_the_whole_dataset (end)
% section clustering_analysis_by_k_means (end)
