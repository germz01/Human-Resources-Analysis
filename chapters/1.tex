\section{Data semantics} % (fold)
\label{sec:data_semantics}
Il progetto viene svolto sul dataset (simulato) \textbf{
\href{https://www.kaggle.com/quentinvincenot/human-resources-analysis/data}{Human Resources Analytics}}. Questo
dataset contiene le informazioni sui dipendenti di un'azienda fittizia, suddivise in base ai seguenti campi:
\begin{center}
    \begin{tabular}{| c | c | m{7cm} |}
    \hline
    \textbf{Field} & \textbf{Type} & \textbf{Description} \\ \hline
    satisfaction\_level & continuous &  valutazione \textit{quantitativa} del livello di soddisfazione di ciascun
    dipendente, ha un valore compreso tra $0$ (minimo) e $1$ (massimo)\\ \hline
    last\_evaluation & continuous & tempo trascorso, in anni, dall'ultima valutazione delle performace del
    dipendente \\ \hline
    number\_project & continuous & numero di progetti completati durante il periodo di lavoro \\ \hline
    average\_montly\_hours & continuous & media delle ore di lavoro in un mese \\ \hline
    time\_spend\_company & continuous & numero di anni trascorsi nell'azienda \\ \hline
    work\_accident & discrete & esprime con un $1$ il coinvolgimento di un dipendente in un incidente sul lavoro,
    altrimenti viene impostato come $0$ \\ \hline
    left & discrete & se il dipendente ha lasciato l'azienda viene impostato come $1$, altrimenti come $0$ \\ \hline
    promotion\_last\_5years & discrete & se il dipendente è stato promosso negli ultimi $5$ anni viene impostato
    come $1$, altrimenti come $0$ \\ \hline
    sales & categorical & definisce il dipartimento nel quale il dipendente lavora \\ \hline
    salary & ordinal & esprime il livello (\textit{low}, \textit{medium}, \textit{high}), di salario nel quale
    rientra il dipendente \\ \hline
\end{tabular}
\end{center}
Per la tipizzazione dei campi presenti nel Dataset è stata seguita la classificazione fornita nel testo
\textit{Guide to Intelligent Data Analysis}, mentre, per la descrizione a parole, sono state seguite le direttive
fornite nella pagina di Kaggle
\footnote{\href{https://www.kaggle.com/}{Kaggle}} nella quale è contenuto il Dataset
\footnote{\href{https://www.kaggle.com/ludobenistant/hr-analytics/data}{Column Metadata}}.
% section data_semantics (end)
\section{Distribution of the variables and statistics} % (fold)
\label{sec:distribution_of_the_variables_and_statistics}

% section distribution_of_the_variables_and_statistics (end)
\section{Assessing data quality} % (fold)
\label{sec:assessing_data_quality}

% section assessing_data_quality (end)
\section{Variable transformations} % (fold)
\label{sec:variable_transformations}

% section variable_transformations (end)
\section{Pairwise correlations and eventual elimination od redundant variable} % (fold)
\label{sec:pairwise_correlations_and_eventual_elimination_od_redundant_variable}

% section pairwise_correlations_and_eventual_elimination_od_redundant_variable (end)
